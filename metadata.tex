%%% Please fill in basic information on your thesis, which will be automatically
%%% inserted at the right places. You need to replace \xxx{...} by real data.

% Type of your thesis:
%	"bc" for Bachelor's
%	"mgr" for Master's
%	"phd" for PhD
%	"rig" for rigorosum
\def\ThesisType{mgr}

% Language of your study programme:
%	"cs" for Czech
%	"en" for English
\def\StudyLanguage{cs}

% Thesis title in English (exactly as in the official assignment)
% (Note: \xxx is a "ToDo label" which makes the unfilled visible. Remove it.)
\def\ThesisTitle{In-memory database for web browser}

% Author of the thesis (you)
\def\ThesisAuthor{Filip Ježek}

% Year when the thesis is submitted
\def\YearSubmitted{2025}

% Name of the department or institute, where the work was officially assigned
% (according to the Organizational Structure of MFF UK in English,
% see https://www.mff.cuni.cz/en/faculty/organizational-structure,
% or a full name of a department outside MFF)
\def\Department{Department of Software Engineering}

% Is it a department (katedra), or an institute (ústav)?
\def\DeptType{Department}

% Thesis supervisor: name, surname and titles
\def\Supervisor{RNDR. Michal Kopecký, Ph.D.}

% Supervisor's department (again according to Organizational structure of MFF)
\def\SupervisorsDepartment{Department of Software Engineering}

% Study programme (does not apply to rigorosum theses)
\def\StudyProgramme{Computer Science - Software and Data Engineering}

% An optional dedication: you can thank whomever you wish (your supervisor,
% consultant, who provided you with tea and pizza, etc.)
\def\Dedication{%
I would like to sincerely thank Michal Kopecký, my supervisor. Similarly, I would like to thank Irena Holubová, Pavel Koupil, and Jáchym Bártík (in no particular order). I very much appreciate the hours spent on consultations, fresh ideas, invaluable feedback, patience, and enthusiasm. I also want to express my gratitude to my friends and family for their unconditional support.
}

% Abstract (recommended length around 80-200 words; this is not a copy of your thesis assignment!)
\def\Abstract{%
This thesis introduces DortDB, a TypeScript in-memory database. DortDB is entirely modular and extensible. It can be configured with different query languages and can combine multiple languages in a single query. To this end, we have designed the unified algebra that supports relational, document, and graph models. We have implemented a query optimizer that optimizes the combined logical plan across language boundaries, and a system for extensible secondary indices. Part of the thesis is the DortDB Showcase, a GUI for inspecting logical plans, experimenting with various optimizations, and querying data. We have evaluated DortDB with other multimodel databases on the UniBench benchmark, and compared it with two JavaScript in-memory SQL databases using the TPC-H queries.
}

% 3 to 5 keywords (recommended) separated by \sep
% Keywords are useful for indexing and searching for the theses by topic.
\def\ThesisKeywords{%
SQL\sep Relational Data Model\sep JSON\sep In-memory database\sep Multimodel database
}

% If any of your metadata strings contains TeX macros, you need to provide
% a plain-text version for use in XMP metadata embedded in the output PDF file.
% If you are not sure, check the generated thesis.xmpdata file.
\def\ThesisAuthorXMP{\ThesisAuthor}
\def\ThesisTitleXMP{\ThesisTitle}
\def\ThesisKeywordsXMP{\ThesisKeywords}
\def\AbstractXMP{\Abstract}

% If your abstracts are long and do not fit in the infopage, you can make the
% fonts a bit smaller by this setting. (Also, you should try to compress your abstract more.)
\def\InfoPageFont{}
%\def\InfoPageFont{\small}  % uncomment to decrease font size

% If you are studing in a Czech programme, you also need to provide metadata in Czech:
% (in English programmes, this is not used anywhere)

\def\ThesisTitleCS{In-memory databáze pro webový prohlížeč}
\def\DepartmentCS{Katedra softwarového inženýrství}
\def\DeptTypeCS{Katedra}
\def\SupervisorsDepartmentCS{Katedra softwarového inženýrství}
\def\StudyProgrammeCS{Informatika - Softwarové a datové inženýrství }

\def\ThesisKeywordsCS{%
SQL\sep Relační model dat\sep JSON\sep In-memory databáze\sep Multimodelová databáze
}

%This thesis introduces DortDB, a TypeScript in-memory database. DortDB is entirely modular and extensible. It can be configured with different query languages and can combine multiple languages in a single query. To this end, we have designed the unified algebra that supports relational, document, and graph models. We have implemented a query optimizer that optimizes the combined logical plan across language boundaries, and a system for extensible secondary indices. Part of the thesis is the DortDB Showcase, a GUI for inspecting logical plans, experimenting with various optimizations, and querying data. We have evaluated DortDB with other multimodel databases on the UniBench benchmark, and compared it with two JavaScript in-memory SQL databases using the TPC-H queries.

\def\AbstractCS{%
Tato práce uvádí DortDB, in-memory databázi implementovanou v TypeScriptu. \mbox{DortDB} je plně modulární a rozšiřitelná. V DortDB lze používat různé dotazovací jazyky a je možné několik jazyků střídat i v rámci jednoho dotazu. Za tímto účelem jsme navrhli sjednocenou algebru, která podporuje relační, dokumentový, i grafový datový model. Implementovali jsme optimizer dotazů, který je díky společné algebře schopen optimalizovat dotazy napříč různými jazyky. Dále jsme navrhli systém rozšiřitelných indexů. Součástí práce je DortDB Showcase, grafické uživatelské rozhraní pro zkoumání logických plánů, experimentaci s různými optimizacemi a dotazování nad daty. DortDB jsme vyhodnotili spolu s jinými multimodelovými databázemi na UniBench testech. Také jsme DortDB srovnali se dvěma dalšími in-memory JavaScriptovými databázemi na dotazech z TPC-H.
}
