\chapter*{Conclusion}
\addcontentsline{toc}{chapter}{Conclusion}

In this thesis, we have designed and implemented DortDB, a TypeScript in-memory database. We have also introduced the unified algebra, which can describe relational, document, and graph queries. The unified algebra allowed us to put into practice the concept of multilanguage queries, where the most suitable or intuitive query languages are used for different parts of the query. The unified algebra is extensible, as we have demonstrated on the XQuery language. This opens the road towards incorporating further data models in the future, if they cannot already be described.

The database itself serves as a framework providing common building blocks and a query environment for separately developed query languages. We have implemented three languages: SQL for the relational model, XQuery for the document model, and Cypher for the graph model. The languages can be further extended with user-defined functions, operators, or aggregates. We have exposed both the abstract syntax trees made of parsed queries and the logical plans as a public API for developers using DortDB. The DortDB is entirely modular and has a small footprint on webpage size when used on the web. The core framework in combination with the SQL language weighs less than 200 KB.

We have also developed a rule-based query optimizer. Thanks to the unified algebra, the optimizer can optimize across language boundaries. It is configurable and extensible by providing further optimization rules. The implemented rules include subquery unnesting rules, selection pushdown, and projection merging. Among other DortDB optimizations is a common interface for implementing secondary indices. The secondary indices can be specialized for various scenarios, be it a range index, a hash index, or a spatial index. The indices themselves decide when they can be used. As proof of concept, we have provided a simple hash table index.

We then demonstrated DortDB's capabilities by matching it against two well-established JavaScript in-memory SQL databases and two well-established multimodel databases. DortDB proved the power of its optimizer by significantly outperforming AlaSQL on the TPC-H benchmark, while coming reasonably close to SQL.js. The comparison with ArangoDB and OrientDB did not go over so well, highlighting gaps in optimizing graph and document queries. Even then, the concepts behind DortDB seem like a viable approach, and the benchmarks helped us identify directions for further development.

The DortDB functionality can be tried out in the Showcase, a GUI we have developed for experimentation with logical plans. Users can query data, inspect the logical plans, and interact with the query optimizer. A demo paper featuring DortDB and DortDB Showcase, coauthored with Pavel Koupil, Michal Kopecký, Jáchym Bártík, and Irena Holubová, was accepted for the upcoming 51\textsuperscript{st} VLDB conference.